
\chapter{Skeleta in Higher Dimensions} \label{chapter4}

Throughout this section, let $k$ be the fraction field of a discrete valuation ring $R$ which has uniformizer $\pi$. 
We will denote by $X$ a smooth, proper, connected $k$-scheme.
Once again, we will use the notion of models to study $k$-analytic spaces, but it is more convenient here to use schemes as opposed to formal schemes.
The aim of this section is to generalise the notion of skeleta to higher dimensions.

\section{$R$-Models}

To begin, we define an \textit{$R$-model} for $X$ to be a flat, proper, $R$-scheme $\model$ equipped with an isomorphism $\model_k \cong X$, where $\model_k = \model \times_{R} k$ denotes the generic fiber.
The assumption that the model is proper may be weakened, but we include it here to simplify the presentation.
A morphism of $R$-models $h: \model' \to \model$ is a morphism of $R$-schemes which is compatible with the isomorphisms $\model'_k \cong X$ and $\model_k \cong X$ in the following sense.
By the universal property of the fiber product, $h$ induces a morphism $h_k: \model'_k \to \model_k$.
We then require that the following diagram commutes, where the diagonal maps are the isomorphisms specified by the model.
\[
\begin{tikzcd}
        \model_k' \arrow[dr, "\sim", sloped] \arrow[rr,"h_k"] && \model_k \arrow[dl, "\sim", sloped] \\
        & X 
\end{tikzcd}    
\]

For an $R$-model $\model$, we may consider the space $\hat{\model}_{\eta}$, which is isomorphic to $\anl{X}$ due to the properness assumption.
We find that $\hat{\model}_{s} \cong \model_s$, and so we obtain a map $\operatorname{red}_{\model} : \hat{\model}_{\eta} \to \model_s$.
Fixing a point $x \in \hat{\model}_{\eta}$ and choosing $U = \spec A \subset \model$ to be such that $x \in \hat{U}_{\eta}$, we find that $\red{\model}{x}$ corresponds to the prime ideal of $A$ given by $\{ a \in A \; \vert \; |a|_x < 1\}$.
We note that if $h: \model' \to \model$ is a morphism of $R$-models, then one sees that $h \circ \operatorname{red}_{\model'} = \operatorname{red}_{\model}$.

Furthermore, $x$ defines a multiplicative seminorm on $\sheaf_{\model, \red{\model}{x}}$.
Indeed, if $f \in \sheaf_{\model, \red{\model}{x}}$, then $f$ is regular on some neighbourhood $U = \spec A$, and we find that $x$ induces a seminorm on $A$.
To see that this is well-defined, let $x \in V \subset U$, with $U = \spec A$ and $V = \spec A'$.
The open immersion $V \to U$ induces an open immersion $\anl{V_k} \to \anl{U_k}$ by \parencite[Proposition 3.4.6]{berk1}, and a morphism $\hat{V}_{\eta} \to \hat{U}_{\eta}$.
By considering the following diagram and replicating the proof of \cref{lemma:affinecaseisaffinoid}, we see that that $\hat{V}_\eta \to \hat{U}_{\eta}$ is the embedding of an affinoid domain.
\[
\begin{tikzcd}
 & \hat{V}_{\eta} \arrow[r, overlay] \arrow[d, overlay] & \hat{U}_{\eta} \arrow[d, overlay] \\
 & \anl{V_k} \arrow[r, overlay] & \anl{U_k}
\end{tikzcd}
\]
It follows from this observation that the following diagram commutes, where the top map is the restriction map, showing that the seminorm is well-defined.
\[
\begin{tikzcd}
        A \arrow[dr, overlay] \arrow[rr,"\operatorname{res}", overlay] && A' \arrow[dl, overlay] \\
        & \resfield 
\end{tikzcd}    
\]

\section{The Skeleton of a Strict Normal Crossings Model}

\subsection{Strict Normal Crossings Divisors}

The notion of a strict normal crossings model can be seen as a generalisation of semistable formal models to the current setting.

\begin{defn}\parencite[Chapter 9, Definition 1.6]{liu}
    Let $D$ be an effective Cartier divisor on a locally Noetherian scheme $X$. Let $\{D_i\}_{i \in I}$ be the irreducible components of $D$ considered as reduced subschemes. Then, the following are equivalent.
    \begin{enumerate}
        \item For each $p \in D$, the ring $\sheaf_{X, p}$ is a regular ring with a regular system of parameters $z_1, \dots, z_n$ such that $D$ is locally given by the vanishing of the monomial $z_1^{N_1} \cdots z_r^{N_r}$ for some $r \leq d$ and positive integers $N_1, \dots, N_r$.
        \item Each $D_i$ is an effective Cartier divisor and the scheme theoretic intersection $\bigcap_{j \in J} D_j$ is a regular subscheme of codimension $|J|$ in $X$, where $J \subset I$ is finite.
    \end{enumerate}
    If $D$ satisfies the above, it is called a \textit{strict normal crossings divisor} on $X$. 
\end{defn}

The special fiber $\model_{s}$ of an $R$-model is a closed subscheme whose ideal sheaf is locally generated by the element $\pi$. 
A module over a discrete valuation ring is flat if and only if the uniformizer $\pi$ is not a zero divisor, hence $\model_{s}$ is an effective Cartier divisor on $\model$.

A regular $R$-model $\model$ where $\model_s$ is a strict normal crossings divisor is called an \textit{snc model}.
For brevity, an \textit{snc system of parameters} will refer to a regular system of parameters at a point $z \in \model_s$ such that $\model_s$ is locally given by the vanishing of the monomial $z_1^{N_1} \cdots z_r^{N_r}$ and $z_i = 0$ is a local equation for the irreducible component $D_i$.
Note that we allow each irreducible component, when considered as a prime divisor, to appear with multiplicity.

In general, it is not known whether snc models exist for a given scheme $X$.
Starting from a proper $R$-model, we may produce an snc model using \textit{resolution of singularities}; here, we state a stronger version which will be useful for some proofs in this chapter.

\begin{theorem}[Embedded Resolution of Singularities] \label{resofsing} \parencite{hironaka, hauserresolution, cossart1, cossart2}
Let $(W, Y)$ be a pair such that $Y$ is a divisor on an excellent, reduced scheme $W$.
Assume $W$ has characteristic zero, or that $W$ has dimension at most three and is separated.
Then, there exists a proper birational morphism $\Pi: W' \to W$ such that $W'$ is regular, the total transform of $Y$ is a strict normal crossings divisor in $W'$ and $\Pi$ is an isomorphism outside of $\operatorname{Sing}(Y) \cup \operatorname{Sing}(W)$, where $\operatorname{Sing}(\cdot)$ denotes the singular locus of a scheme.
\end{theorem}

For our purposes, the algebraic condition of excellence always holds since any complete local Noetherian ring is excellent, and a finitely generated algebra over an excellent ring is excellent.
Proper $R$-models exist due to Nagata's compactification theorem, and then \Cref{resofsing} may be used, assuming that $X$ satisfies the relevant conditions, to produce an snc model from a proper $R$-model $\model$, by applying the theorem to the pair $(\model, \model_s)$.
In general, we will assume the existence of an snc model for $X$.

To begin, we explore how blow-ups allow us to construct snc models from existing ones.
Throughout the chapter, we will freely use the following standard facts about blow-ups, references for which can be found in \parencite[Chapter 8]{liu}.

Firstly, if $X$ is a scheme and $\Pi: X' \to X$ is the blow-up along some center $Z$, then for any open $U \subset X$, $\Pi^{-1}(U)$ is isomorphic to the blow-up of $U$ along $U \cap Z$.
Hence, blow-ups can be computed locally and patched together.

Assume next that $U = \spec A$ is a Noetherian, integral affine scheme and the blow-up $U' \to U$ has center cut out by the ideal $I = (f_1, \dots, f_n)$, where $f_i \neq 0$.
Then, $U'$ is covered by affines of the form $\spec A_i$, where $A_i \subset \operatorname{Frac}(A)$ is the $A$ subalgebra of $\operatorname{Frac}(A)$ generated by $f_j/f_i$, for $j \neq i$.
It follows from these facts that if $X$ is additionally a flat $R$-scheme, then $X'$ is also a flat $R$-scheme.
If both $X$ and the center $Z$ are regular, then the blow-up of $X$ along $Z$ is also regular.

If $X$ is locally Noetherian, the blow-up along a closed subscheme cut out by a quasi-coherent ideal sheaf is a proper morphism, and $\Pi$ restricts to an isomorphism away from the center of the blow-up.
Furthermore, a composition of proper morphisms is proper, hence a sequence of blow-ups gives a a proper morphism.

\begin{prop}\label{lemma:blowupsnc}
    Let $\model$ be an snc model and $Z$ a closed subscheme such that one of the following holds:
    \begin{enumerate}
        \item $Z = \{z\}$ for a closed point $z$, such that $z \in \model_s$;
        \item $Z$ is a connected component of an intersection $\cap_{i = 1}^{r} D_i$ of irreducible components of $\model_s$.
    \end{enumerate}
    Let $\Pi: \model' \to \model$ be the blow-up with center $Z$. Then $\model'$ is an snc model.
\end{prop}
\begin{proof}
    Assume we are in the first case.
    There is an isomorphism of the generic fibers $\model'_k \cong \model_k$ since the blow-up has center contained in the special fiber.
    We see that $\model'$ is a regular $R$-model by standard results on blow-ups. 
    
    We now show that $\model'_s = \model^{*}_s + E$ is an snc divisor, where $\model^{*}_s$ denotes the strict transform of $\model_s$ and $E$ is the exceptional divisor. 
    It suffices to work in an affine neighbourhood $U = \spec A$ of $z$, in which case the blow-up $U' \to U$ with center $z$ may be computed as follows. 
    Let $z_1, \dots, z_n$ be an snc system of parameters at $z$, so that by shrinking $U$ as necessary, $z$ corresponds to a maximal ideal of $A$ generated by $(z_1, \dots, z_n)$ and $\model_s$ is given by the vanishing of the function $z_1^{N_1} \cdots z_r^{N_r}$ in $U$.
    The blow-up $U' \to U$ can be covered by charts $U_i = {\spec A_i}$ where 
    \[A_i := A[T_1, \dots, \hat{T_i}, \dots T_n]/(T_j z_i - z_j)_{j \neq i}.\] 
    We show that $\model'_{s} \cap U_i$ is a divisor with strict normal crossings in each $U_i$.
    
    Consider the case where $i = r$; the other cases are similar. 
    Then, if $\mathfrak{J}$ is the ideal of $A$ generated by $\pi$, we have
    \begin{align*}
        \mathfrak{J} \cdot A_i & = (z_1^{N_1} \cdots z_r^{N_r}) \cdot A_i
        = \left(\left(\frac{z_1}{z_r}\right)^{N_1} \cdots \left(\frac{z_{r-1}}{z_r}\right)^{N_{r - 1}} \cdot z_r^{N_1 + \dots + N_r} \right)
        = \left(T_1^{N_1} \cdots T_r^{N_r} \cdot z_r^{N_1 + \dots + N_r} \right).
    \end{align*}
    Since the exceptional divisor is given by the vanishing of $z_r$ in the chart $U_r$, it follows that the strict transform $\model^{*}_{s}$ is cut out by the monomial $T_1^{N_1} \cdots T_{r - 1}^{N_{r -1}}$ in $U_r$.
    A prime component is given by the vanishing of $T_j$, and hence is the strict transform of the prime component $D_j$ of $\model_s$ given by the vanishing of $z_j$.
    Since the strict transform of $D_j$ is isomorphic to the blow-up of $D_j$ with center $z$, we see that it is regular.
    To show regularity of the intersections, it suffices to consider the subscheme $\spec A_i/(T_{i_1}, \dots, T_{i_m}, z_r)$ for some set of indices $i_1, \dots, i_m$.
    We assume for ease of notation that the indices are $\{1, \dots, m\}$, for some $m < r$.
    We now find that:
    \begin{align*}
        & \spec A_i/(T_1, \dots, T_m, z_r) \\ 
        & \cong \spec A[T_1, \dots, \hat{T_r}, \dots, T_n]/(T_jz_r - z_j, T_1, \dots, T_m, z_r)_{j \neq r} \\
        & \cong \spec (A/(z_1, \dots, z_n))[T_{m + 1}, \dots, T_n] \\
        & \cong \spec K[T_{m + 1}, \dots, \hat{T_r}, \dots, T_n].
    \end{align*}
    where $K$ is a field.
    Hence, we see that the intersection is regular and of codimension $n - (n - m - 1) = m + 1$, as required.
    
    The proof of the second case is done via a similar computation.
\end{proof}

\subsection{Monomial and Divisorial Points}

Let $\model$ be an snc model for $X$.
Using the notation as above, if $\xi$ is a generic point of some connected component of an intersection $\cap_{i = 1}^{r} D_i$ of $\model_s$, then there exists an snc system of parameters $z_1, \dots, z_r$ and a unit $u$ in $\sheaf_{\model, \xi}$ such that $\pi = u z_1^{N_1} \cdots z_r^{N_r}$ for some positive integers $N_1, \dots, N_r$. 
The aim of this section is to see how the special fiber identifies a subset of points of $\hat{\model}_{\eta}$, known as \textit{monomial and divisorial} points.

To begin the construction of such points, we require the following lemma, which can be seen as a generalisation of the idea that for a local ring $A$ with a principal maximal ideal $\maxideal = (\pi)$, every element of the $\maxideal$-adic completion $\hat{A}$ can be written as a power series in $\pi$.

\begin{lemma}\parencite[Lemma 2.4.4]{MN}
    Let $z_1, \dots, z_n$ be a regular system of parameters for the maximal ideal of $\sheaf_{\model, \xi}$. 
    Let $\hat{\sheaf}_{\model, \xi}$ be the completion with respect to the maximal ideal. 
    Then, every element $f \in \hat{\sheaf}_{\model, \xi}$ can be written in the form
    \begin{equation}\label{eqn:admissibleexpansion}
        f = \sum_{\beta \in \mathbb{Z}^n_{\geq 0}} c_{\beta} z^{\beta}
    \end{equation}
    where each $c_{\beta}$ is zero or a unit in $\hat{\sheaf}_{\model, \xi}$.
    Such an expansion is called an \textit{admissible expansion} for $f$.
\end{lemma}
\begin{proof}
    The proof is reproduced from \textit{loc. cit.} here.
    Denote by $\maxideal_{\xi}$ the maximal ideal of $\hat{\sheaf}_{\model, \xi}$.
    Fix an element $f \in \hat{\sheaf}_{\model, \xi}$ and assume that $f$ lies in $\maxideal_{\xi}$ since otherwise we are done. 
    Now let $i \geq 1$ and assume that for all $j \leq i$, any $f' \in \hat{\sheaf}_{\model, \xi}$ has an expansion of the form:
    \[
        f' = f'_i + \left( \sum_{|\beta| = i} c_\beta z^{\beta} \right)
    \]
    where $f'_i$ is an element which has an admissible expansion and each $c_\beta$ is some element of $\hat{\sheaf}_{\model, \xi}$. 
    In particular, for $j = i$, $f$ has such an expansion:
    \[
        f = f_i + \left( \sum_{|\beta| = i} c_\beta z^{\beta} \right).
    \]
    For each coefficient $c_\beta$ in the expansion for $f$, we may apply the assumption with $j = 1$, so that we can write $f$ in the form:
    \begin{align*}
        f & = f_{i} + \sum_{|\beta| = i} (c_{\beta, i} + d_{1} z_1 + \dots + d_n z_n) z^{\beta} \\
        & = \left(f_i + \sum_{|\beta| = i} c_{\beta, i} z^{\beta}\right) + \left(\sum_{|\beta| = i} d_1 z_1 z^{\beta} + \dots + d_n z_n z^{\beta}\right) \\
        & = f_{i + 1} + g_{i + 1}
    \end{align*}
    where $c_{\beta, i}$ is an element admitting an expansion of the form \ref{eqn:admissibleexpansion}.
    We see that $f_{i + 1}$ is an element with an admissible expansion while $g_{i + 1}$ is a monominal in $z_1, \dots, z_n$ of degree $i + 1$.
    Furthermore, the coefficients of $f_{i + 1}$ and $f_{i}$ agree in degree $ < i$.
    Iterating this gives an admissible expansion for $f$.
\end{proof}

\begin{lemma}\label{lemma:zerouniqueexpansion} \parencite[Prop. 2.4.6]{MN}
    The element $0 \in \hat{\sheaf}_{\model, \xi}$ has a unique admissible expansion, where all coefficients are $0$.
\end{lemma}
\begin{proof}
    Let
    \[
        0 = \sum_{\beta \in \ZZ^{r}_{\geq 0}} d_{\beta} z^{\beta}
    \]
    be an admissible expansion for $0$.
    The claim is that for each $\beta$, $d_{\beta}$ lies in the maximal ideal $\maxideal$ of $\hat{\sheaf}_{\model, \xi}$, hence must be zero.
    We show this by inducting on $|\beta|$, and using the following result from commutative algebra.
    \begin{lemma}\parencite[Prop. 11.20]{AM}
        Let $z_1, \dots, z_r$ be a regular system of parameters for a local ring $A$ with maximal ideal $\maxideal$ and $f(t_1, \dots, t_r)$ a homogeneous polynomial of degree $s$ with coefficients in $A$. 
        Assume that $f(z_1, \dots, z_r) \in \maxideal^{s + 1}$. 
        Then all coefficients of $f$ lie in $\maxideal$.
    \end{lemma}
    In the base case, we have that $d_{0}$ must lie in the maximal ideal, since
    \[
        d_{0} = - \sum_{|\beta| > 1} d_{\beta} z^{\beta}.
    \]
    Assuming that $d_{\beta} = 0$ for all $\beta$ such that $|\beta| \leq s$ for some $s \geq 0$, we consider the homogeneous polynomial of degree $s + 1$ given by:
    \[
        f = \sum_{|\beta| = s + 1} d_{\beta} t^{\beta}.
    \]
    The parameters $z_1, \dots, z_r$ generate $\maxideal$ and we must have that $f(z_1, \dots, z_r)$ lies in $\maxideal^{s + 1}$ by the inductive hypothesis and rearranging the expansion.
    Hence, all coefficients lie in $\maxideal$ by the lemma.
\end{proof}

The utility of an admissible expansion is the following.
Firstly, note that for any $f \in \sheaf_{\model, \red{\model}{x}}$ for a point $x \in \hat{\model}_{\eta}$, $v_x(f) \geq 0$. 
From this we observe that if $f$ is a unit in $\sheaf_{\model, \red{\model}{x}}$, then $v_x(f) = -v_x(f^{-1})$, so $v_x(f) = 0$.
The same then holds true for $\hat{\sheaf}_{\model, \red{\model}{x}}$.
Hence, finding an admissible expansion for an element $f \in \sheaf_{\model, \red{\model}{x}}$ as in \cref{eqn:admissibleexpansion}, we see that 
\begin{equation}\label{equationforx}
    v_x(f) \geq \min \{ \alpha \cdot \beta \; \vert \; \beta \in \ZZ_{\geq 0}^{n}, c_{\beta} \neq 0 \}
\end{equation}
where $\alpha \cdot \beta$ denotes the dot product, and $\alpha = (v_x(z_1), \dots, v_x(z_n))$.
Due to the equation $\pi = u z_1^{N_1} \cdots z_r^{N_r}$, we must have that $v_x(z_1) N_1 + \cdots v_x(z_r) N_r = 1$.
These ideas lead to the construction of `monomial points', which are the points where \cref{equationforx} is an equality.

\begin{prop}\label{monomialpoint} \parencite[Prop. 2.4.6]{MN}
    Let $D_1, \dots, D_r$ be irreducible components of $\model_s$, and $\xi$ the generic point of a connected component of $\cap_{i = 1}^{r} D_i$.
    Let $N_i$ be the multiplicity of $D_i$ in $\model_s$ and let $\alpha = (\alpha_1, \dots, \alpha_r) \in \RR^{r}_{\geq 0}$ be such that $\alpha_1 N_1 + \dots + \alpha_r N_r = 1$.
    Then, there exists a point $x \in \hat{\model}_{\eta}$ inducing a valuation $v_{x}: K(X)^{\times} \to \RR$ such that for each $f \in \sheaf_{\model, \xi}$ and any admissible expansion of $f$ as in \cref{eqn:admissibleexpansion}, we have:
    \[
        v_{x}(f) = \min \{ \alpha \cdot \beta \; \vert \; \beta \in \mathbb{Z}_{\geq 0}^m, c_{\beta} \neq 0\}.
    \]
     The point $x$ is then called a \textit{monomial point} with presentation $(\model, (D_1, \dots, D_r), \xi, \alpha)$.
\end{prop}
\begin{proof}
    We give an alternative proof from the one presented in \textit{loc. cit.}.

Firstly, assume by permuting indices that $\alpha = (\alpha_1, \dots, \alpha_{r'}, 0, \dots, 0)$, where $\alpha_1, \dots, \alpha_{r'}$ are non-zero, for some $r' \leq r$.

Define for each $\alpha \in \mathbb{R}^r_{\geq 0}$ and $w \in \RR$ the ideal $I_{\geq w}$ of $\hat{\sheaf}_{\model, \xi}$ generated by the monomials $z^{\beta}$ such that $\alpha \cdot \beta \geq w$, and the ideal $I_{> w}$ generated by the monomials $z^{\beta}$ such that $\alpha \cdot \beta > w$.
Denoting by $\kappa$ the residue field of $\hat{\sheaf}_{\model, \xi}$, the claim is that $I_{\geq w} / I_{> w}$ is a $\kappa$ vector space with a basis given by $z^{\beta}$ where $\alpha \cdot \beta = w$.
To see this, first note that if $\alpha \cdot \beta = w$ for some $\beta$ and $w$, then $z^{\beta} \in I_{\geq w} \backslash I_{> w}$.
Indeed, suppose that $z^{\beta} \in I_{> w}$; then, we may write $z^{\beta} = c_1 z^{\beta_1} + \cdots + c_n z^{\beta_n}$, and, choosing admissible expansions for each $c_i$, we may find an admissible expansion
\[
    z^{\beta} + \sum_{\alpha \cdot \gamma > w} c_{\gamma} z^{\gamma} = 0.
\]
By, \cref{lemma:zerouniqueexpansion}, each coefficient in the admissible expansion must be zero, giving a contradiction.
Hence, the classes of the elements $z^{\beta}$ give a generating set for $I_{\geq w} / I_{> w}$.
A similar argument shows that any finite sum \[ c_1 z^{\beta_1} + \dots + c_m z^{\beta_m} \] where each $c_i$ is a unit, is not contained in $I_{> w}$, so that the classes of the elements $z^{\beta}$ with $\alpha \cdot \beta = w$ form a basis for $I_{\geq w}/I_{> w}$.

For each element $f$ of $\hat{\sheaf}_{\model, \xi}$, there exists some $w$ such that $f \in I_{\geq w} \backslash I_{> w}$, which must be unique since $\RR$ is totally ordered.
To see this, fix an admissible expansion for $f$ with coefficients $c_{\beta}$ and let $v = \min \{ \alpha \cdot \beta \; | \; c_{\beta} \neq 0 \}$. 
We can then write
\begin{align*}
    f & = \sum_{\alpha \cdot \beta = v} c_{\beta} z^{\beta} + \sum_{\alpha \cdot \beta > v} c_{\beta} z^{\beta} \\
    & = f' + f''.
\end{align*}
The ideal $I_{> v}$ is closed in the $\maxideal_{\xi}$-adic topology; for this it suffices to show that
\[
    \bigcap_{i > 0} (I_{> v} + \maxideal_{\xi}^{i}) \subset I_{> v}.
\]
This follows from the fact that $\maxideal^{i}_{\xi}$ is generated by monomials of degree $i$ in the $z_1, \dots, z_n$, so  for $i$ large enough, $\maxideal^{i}_{\xi} \subset I_{> w}$.
A similar argument shows that $I_{\geq v}$ is closed.
It follows from this that $f'' \in I_{> v}$ and $f' \in I_{\geq v}$, and by considering reductions modulo $I_{> v}$ we see that $f \not\in I_{> v}$.
Hence $v$ is the unique real number such that $f \in I_{\geq v} \backslash I_{> v}$, and in particular $v_x$ is independent of the choice of admissible expansion.

Since ideals are closed under addition, we find that $v_x(f + g) \geq \min \{ v_x(f), v_x(g) \}$.
Next, let $\overline{f}$ be the reduction of $f$ modulo $I_{> v}$, and write
\[
    \overline{f} = \sum_{\alpha \cdot \beta = w} d_{\beta} z^{\beta}
\]
where $d_{\beta} \in \kappa$.
We define $\Gamma_f$ as follows: for each choice of $\beta_1, \cdots, \beta_r$ such that $\alpha_1 \beta_1 + \dots + \alpha_{r'} \beta_{r'} = v$, $\beta = (\beta_1, \cdots, \beta_r)$ is in $\Gamma_f$ if $\beta_{r' + 1} + \dots + \beta_{r}$ is minimal amongst all possible choices of $\beta_{r' +1}, \dots, \beta_{r}$ such that $d_{\beta} \neq 0$.
We then define the polynomial
\[
    f_{\alpha} = \sum_{\beta \in \Gamma_f} d_{\beta} z^{\beta}.
\]
Then, we can view $f_{\alpha}$ as an element of $\kappa[z_1, \dots, z_n]$.
To show $v_x(f \cdot g) = v_x(f) + v_x(g)$, we now fix admissible expansions for $f$ and $g$, and a computation shows that $(f \cdot g)_{\alpha} = f_{\alpha} \cdot g_{\alpha}$, where the latter product is taken in the ring $\kappa[z_1, \dots, z_n]$.

Since $\pi = u \cdot z_1^{N_1} \cdots z_r^{N_r}$ is an admissible expansion, $v_x(\pi) = \alpha_1 N_1 + \cdots + \alpha_r N_r = 1$. 
Hence, $v_x$ extends the valuation on $k$. 
We conclude by noting that although $v_x$ is currently only defined on the local ring $\sheaf_{\model, \xi}$, it extends to a valuation on $K(\model) \cong K(X)$ by setting $v_x(f/g) = v_x(f) - v_x(g)$.
\end{proof}

Let $\xi$ be the generic point of a connected component of $\cap_{i = 1}^{r} D_r$, and suppose $\alpha$ has zero entries.
Then we may, after permuting indicies, assume that $\alpha = (\alpha_1, \dots, \alpha_{r'}, 0, \dots, 0)$ with $\alpha_1, \dots, \alpha_{r'}$ non-zero.
Then, the formula for $v_x$ shows that it is also the monomial point with presentation \[(\model, (D_1, \dots, D_{r'}), \xi', \alpha')\] where $\alpha' = (\alpha_1, \dots, \alpha_{r'})$ and $\xi'$ is the generic point of the connected component of $\cap_{i = 1}^{r'} D_i$ which contains $\xi$.

When $\xi$ is the generic point of an irreducible component $D$, then $\sheaf_{\model, \xi}$ is a discrete valuation ring with fraction field equal to $K(X)$. 
In this case, the valuation on $K(X)$ such that the valuation ring is $\sheaf_{\model, \xi}$ coincides with the valuation given by the monomial point corresponding to the data $(\model, D, \xi, \alpha = 1)$.
The point is then called a \textit{divisorial point} associated to the data $(\model, D)$.

Note that a point which is monomial with respect to a given model may in fact be divisorial with respect to another snc model. 
The following lemma and proposition form the contents of the proof of \parencite[Prop. 2.4.11]{MN}, and characterize when such a case occurs.

\begin{lemma} \label{blowupintersection}
    Let $D_1, \dots, D_r$ be prime components of $\model_s$ for some snc model $\model$ and $\xi$ the generic point of a connected component of $D_1 \cap \dots \cap D_r$. 
    Let $\model' \to \model$ be the blow-up at the closure of $\xi$.
    If $x \in \hat{\model}_{\eta}$ is a point with monomial presentation $(\model, (D_1, \dots, D_r), \xi, \alpha)$, then $x$ is also monomial with respect to $\model'$.
    If $\alpha = (\alpha_1, \dots, \alpha_r)$ is such that $\alpha_1$ is the minimal coordinate, $D'_2, \dots, D'_r$ are the strict transforms of $D_2, \dots, D_r$ and $E$ is the exceptional divisor, then $x$ has monomial presentation $(\model', (E, D'_2, \dots, D'_r), \xi', \alpha')$ where $\xi'$ is the generic point of $E \cap D'_2 \cap \dots \cap D'_r$  and $\alpha'$ is the tuple
    \[
        \alpha' = (\alpha_1, \alpha_2 - \alpha_1, \dots, \alpha_r - \alpha_1).
    \]
\end{lemma}
\begin{proof}
    By \cref{lemma:blowupsnc}, $\model'$ is an snc model.
    Let $U = \spec A$ be an affine neighbourhood of $\xi$ such that if $z_1, \dots, z_r$ is an snc system of parameters, then $\xi$ corresponds to the prime ideal $(z_1, \dots, z_r)$ of $A$.
    We have that $\red{\model'}{x}$ is the point of $\spec A[T_2, \dots, T_r]/(T_jz_1 - z_j) \subset \model'$ corresponding to the prime ideal $(z_1, T_2, \dots, T_r)$.
    The vanishing of the elements $T_j$ define the strict transforms $D'_j$ of $D_j$ for each $2 \leq j \leq r$, and the vanishing of $z_1$ defines the exceptional divisor.
    In particular, we can take $z_1, T_2, \dots, T_r$ to be an snc system of parameters.
    
    Now, let $x'$ be the monomial point corresponding to the data $(\model', (E, D'_2, \dots, D'_r), \xi', \alpha')$ as in the statement of the lemma.
    Since $\red{\model}{x'} = \red{\model}{x}$, to show that $x' = x$, it suffices to show that for all $f \in \sheaf_{\model, \red{\model}{x}}$, $v_x(f) = v_{x'}(f)$.
    
    Fix $f \in \sheaf_{\model, \red{\model}{x}}$.
    Let
    \[
        f = \sum c_{\beta} z_1^{\beta_1} \cdots z_r^{\beta_r}
    \]
    be an admissible expansion for $f$ in $\hat{\sheaf}_{\model, \red{\model}{x}}$.
    We note that 
    \[
        f = \sum c_{\beta} z_1^{\beta_1 + \dots + \beta_r} T_2^{\beta_2} \cdots T_r^{\beta_r}
    \]
    gives an admissible expansion for $f$ in $\hat{\sheaf}_{\model, \red{\model'}{x'}}$.
    It follows immediately from the formulae for $v_x$ and $v_{x'}$ that $v_x(f) = v_{x'}(f)$.
 \end{proof}

\begin{prop} \label{monomialisdivisorial} \parencite[Proposition 2.4.11]{MN}
    A monomial point $x$ of $\hat{\model}_{\eta}$ with a monomial presentation \[(\model, (D_1, \dots, D_r), \xi, \alpha)\] is divisorial if the entries of $\alpha$ are rational.
\end{prop}
\begin{proof}
    Let $\model' \to \model$ be the blow-up at the closure of $\xi$.
    We may assume by permuting indices that $\alpha_1$ is the minimial coordinate. Then, by \cref{blowupintersection}, $\model'$ is an snc model where the monomial presentation of $x$ with respect to $\model'$ is such that $\alpha' = (\alpha_1, \alpha_2 - \alpha_1, \dots, \alpha_r - \alpha_1)$.
    Since each coordinate is rational, we can assume $\alpha$ is of the form $\left(\frac{a_1}{b}, \dots, \frac{a_r}{b} \right)$ for positive integers $a_1, \dots, a_r$ and some integer $b$.
    Hence, repeating this process and removing any zero entries, we reduce to the case where $r = 1$.
    This gives a divisorial presentation.
\end{proof}

\subsection{The Retraction to the Skeleton}

The skeleton $\sk{\model}$ of $\anl X$ induced by an snc model $\model$ is defined as the set of points of $\anl X$ which are monomial with respect to $\model$.
The space $\sk{\model}$ has a particularly elegant topological description as the \textit{dual complex} $\Delta(\model_s)$ of $\model_s$ \parencite[\S 2.4.2]{birgeom}.

Let $\{D_i\}_{i \in I}$ be the irreducible components of $\model_s$, and $J \subset I$ a non-empty finite subset.
Then, the dual graph $\Delta(\model_s)$ contains an $r$-simplex for each connected component of $\cap_{j \in J} D_j$, where $r = |J| - 1$.
We make the following identifications. 
Let $C \subset \cap_{j \in J} D_j$ and $C' \subset \cap_{k \in J'} D_k$ be two connected components corresponding to simplicies $E$ and $E'$. 
Then $E$ is a face of $E'$ if and only if we have $C' \subset C$.
There is a homeomorphism $\Phi: \Delta(\model_s) \to \sk{\model}$.
We describe the construction of the map and omit the proof that it gives a homeomorphism which may be found in \parencite[\S 2.4.4]{birgeom}.
Let $x \in \Delta(\model_s)$ be a point; if it is a vertex, then it corresponds to an irreducible component $D \subset \model_s$ and we map it to the divisorial point with presentation $(\model, D)$.
Otherwise, $y$ lies on the interior of a simplex corresponding to some connected component of an intersection $\cap _{j \in J} D_j$.
Let $r = |J|$, $\xi$ be the generic point of the component and let $(w_1, \dots, w_{|J|})$ be the coordinates of $y$ considered as a point on the simplex.
Then $w_1 + \dots + w_{r} = 1$, so $\alpha = \left({w_1}/{N_1}, \dots, {w_r}/{N_r} \right)$ is such that $\alpha_1 N_1 + \dots + \alpha_r N_r = 1$.
We then map $y$ to the monomial point with presentation $(\model, (D_1, \dots, D_r), \xi, \alpha)$.

The inclusion $\sk{\model} \subset \hat{\model}_{\eta}$ admits a continuous retraction $\rho_{\model} : \hat{\model}_{\eta} \to \sk{\model}$ \parencite[\S 3.1.5]{MN}. 
For any $x$, let $D_1, \dots, D_r$ be the irreducible components of $\model_s$ passing through $\red{\model}{x}$. 
Fix $1 \leq i \leq r$.
Letting $z_1, \dots, z_n$ be an snc system of parameters at $\red{\model}{x}$ such that $D_i$ is locally given by the vanishing of $z_i$, set $v_x(D_i) := v_x(z_i)$.
Note that $N_1v_x(z_1) + \cdots + N_r v_x(z_r) = v_x(\pi) = 1$, so that setting $\alpha = (v_x(z_1), \dots, v_x(z_r))$ and letting $\xi$ be the generic point of the connected component of $\cap_{i = 1}^r D_i$ which contains $\red{\model}{x}$, we set $\rho_{\model}(x)$ to be the point with monomial presentation $(\xi, \alpha)$ with respect to the model $\model$.
This is well-defined: the parameters $z_1, \dots, z_r$ are determined up to multiplication by a unit in $\sheaf_{\model, \red{\model}{\xi}}$, so if $z'_i$ is another function defining $D_i$ locally, then $z_i = u \cdot z'_i$ for some unit $u \in \sheaf_{\model, \red{\model}{\xi}}$.
Since $v_x(u) = 0$ it follows that $v_x(z_i) = v_x(z'_i)$.
We also see that $v_x(D) = v_{\rho_{\model}(x)}(D)$ for every prime component $D$ of $\model_s$ passing through $\red{\model}{x}$.
We remark for later use that the construction $v_x(D)$ extends to any divisor $D$ which is not supported on $\iota(x)$, by similarly choosing a local equation for $D$ at $\red{\model}{x}$.

The retraction map has the following properties, which we state without proof. 
Denote $y = \rho_{\model}(x)$, for any $f \in \sheaf_{\model, \red{\model}{x}}$. 
If $f \in \sheaf_{\model, \red{\model}{x}}$ is a function regular on some neighbourhood $U$ of $\red{\model}{x}$, then note that $U$ contains $\red{\model}{y}$ so that we may consider $|f|_y$. 
Then, $|f|_x \leq |f|_y$ for every $f \in \sheaf_{\model, \red{\model}{x}}$ \parencite[Prop. 3.1.6]{MN}. 
The following proposition shows that the retraction map is well-behaved with respect to blow-ups.

\begin{prop} \label{retractionsinversesystem} \parencite[Prop. 3.1.7]{MN}
    If $\model' \to \model$ is a proper morphism of snc models of $X$, i.e. a proper morphism of $R$-models where both $\model, \model'$ are snc, then $\rho_{\model} \circ \rho_{\model'} = \rho_{\model}$ and $\sk{\model} \subset \sk{\model'}$.
\end{prop}

In the case of curves, we have the following characterisation of the retraction points.

\begin{prop}\label{lemma:retractiondivisorial}
    Let $X$ be a curve and $x, y \in \hat{\model}_{\eta}$ be distinct points such that $\rho_{\model}(x) = \rho_{\model}(y)$. 
    Then, $\rho_{\model}(x)$ is a divisorial point.
\end{prop}
\begin{proof}
    Without loss of generality, we may assume that $y = \rho_{\model}(x)$ is a monomial point which is not divisorial with respect to $\model$.
    Let $(\xi, (\alpha_1, \alpha_2))$ be a monomial presentation for $y$ with respect to the model $\model$ and let $z_1, z_2$ be a regular system of parameters at $\red{\model}{x} = \red{\model}{y} =: \xi$, such that $\xi$ is the generic point of the intersection of the vanishing loci of $z_1$ and $z_2$. 
    It suffices to show that $|z_1^a|_y = |\pi^b|$ for some $a, b$, since then $\alpha_1$ and $\alpha_2$ must be rational and we can conclude using \cref{monomialisdivisorial}.
    
    Assume $x$ is not a type III point; then we are done, since $|\resfield^{\times}|/|k^{\times}|$ is a rank $0$ abelian group and hence for every element $r \in |\resfield^{\times}|$, we have $r^a \in |k^{\times}|$. for some $a \in \ZZ_{\geq 0}$
    
    Now, we assume that both $x$ and $y$ are type III points and proceed to assume that $\alpha_1$ and $\alpha_2$ are irrational.
    Since $x$ and $y$ are distinct, we can assume that there exists some $f \in \sheaf_{\model, \red{\model}{x}}$ such that $v_x(f) \neq v_y(f)$. 
    Choosing an admissible expansion for $f$:
    \[
        f = \sum c_{ab} z_1^{a} z_2^{b}
    \]
    we see that
    \begin{align*}
        v_x(f) &= v_x \left(\sum c_{ab} z_1^{a} z_2^{b} \right) \\
              &= \min \{ v_x(c_{ab}) + aN_1 \alpha_1 + bN_2 \alpha_2 \} \\
              &= \min \{ v_y(c_{ab}) + aN_1 \alpha_1 + bN_2 \alpha_2 \} \\
              &= v_x \left(\sum c_{ab} z_1^{a} z_2^{b} \right) \\
              & = v_y(f)
    \end{align*}
    where we have used the irrationality of $\alpha_1$ and $\alpha_2$ to sharpen the triangle inequality into an equality, and the fact that $v_x(c) = v_y(c) = 0$ for any unit $c \in \hat{\sheaf}_{\model, \red{\model}{x}}$.
    Hence, we arrive at a contradiction, so it follows that $\alpha_1$ and $\alpha_2$ are rational. 
\end{proof}

A corollary of the proof of \cref{lemma:retractiondivisorial} is that for any type III point $x$ monomial with respect to an snc model $\model$, we have that $\rho_{\model}^{-1}(x) = \{x\}$. 

We now consider how blow-ups of snc models affect the skeleton of a curve.
Let $\model$ be an snc model for a curve $X$, $D_1, D_2$ irreducible components of $\model_s$ and $z$ the closed point contained in a connected component of the intersection $D_1 \cap D_2$.
Let $\model' \to \model$ be the blow-up at $z$ and $\sigma$ the interval of $\sk{\model}$ corresponding to $z$.
Then, the simplicial structure of $\sk{\model'}$ is given by barycentrically subdividing $\sigma$ in $\sk{\model'}$.

To see this, let $E$ be the exceptional divisor and $D'_1$, $D'_2$ the strict transforms of $D_1$ and $D_2$ respectively. 
Firstly, if $\red{\model}{x}$ is the generic point of $E$, then it follows from \cref{blowupintersection} that 
    \[(\model, (D_1, D_2), z, (1/(N_1 + N_2), 1/(N_1 + N_2))\] 
is a monomial presentation for $x$. 
If $\red{\model}{x}$ is, without loss of generality, the generic point of $D'_1$, then $(\model, D_1)$ is a divisorial presentation for $x$.

Next, we consider a point $x$ such that $\red{\model'}{x} =: \xi$ is the generic point of an intersection $E \cap D'_2$, after possibly permuting the induces.
Hence, let $(\model', (E, D_2), \xi, \alpha)$ be a monomial presentation for $x$.
In this case, it follows from \cref{blowupintersection} that 
\[(\model, (D_1, D_2), z, (\alpha_1, \alpha_2 + \alpha_1))\] is a monomial presentation for $x$.
    
Hence, we have shown that $\sk{\model'} \subset \sk{\model}$, from which it follows that $\sk{\model'} = \sk{\model}$. 
Considering the simplicial structure, there is an extra vertex corresponding to the exceptional divisor, which is located in the barycenter of the $1$-simplex corresponding to the point $z$.

This is in fact a special case of the following more general theorem, which we present without proof. 

\begin{prop}\label{blowupintersectionskeleton}\parencite[Prop. 3.1.9]{MN}
    Let $D_1, \dots, D_r$ be irreducible components of $\model_s$, $\xi$ the generic point of a connected component of the intersection $D_1 \cap \dots \cap D_r$ and $\sigma$ the face of $\sk{\model}$ corresponding to $\xi$. If $\model' \to \model$ is the blow-up with center given by the closure of $\xi$, then $\sk{\model'} = \sk{\model}$, and the simplicial structure of $\sk{\model'}$ is obtained by adding a vertex to the barycenter of $\sigma$ and joining it to the faces of $\sigma$.
\end{prop}

Now let $\model$ be an snc model for a curve $X$, $D$ an irreducible component of $\model_s$ and $z$ a closed point on $D$ such that no other irreducible components pass through $z$.
Let $\model' \to \model$ be the blow-up centered at $z$.
In this case, $\sk{\model}$ is a strict subset of $\sk{\model'}$.
Consider a point $x \in \sk{\model'}$ such that the closure of $\red{\model'}{x}$ is contained in the exceptional divisor $E$. 
We have that $\red{\model}{x} = z$, hence $x$ is not monomial with respect to $\model$.
We can see that the simplicial structure of $\sk{\model'}$ is obtained by adjoining a line segment to a vertex of $\sk{\model}$: more precisely, the vertices of the line segment correspond to the divisorial points associated with $E$ and $D$, while the interior corresponds to the monomial points with presentation $(\model', (E, D), \xi, \alpha)$, for some $\alpha$.

Finally, we note the following proposition due to Berkovich and Thuillier shows that for an snc model $\model$, $\sk{\model}$ is a strong deformation retract of $\anl X$.

\begin{prop} \parencite[Theorem 3.26]{thuillier}
    There exists a continuous map 
    \[
        H: \hat{\model}_\eta \times [0, 1] \to \hat{\model}_\eta
    \]
    such that $H_0(x) = H(x, 0)$ is the identity map, $H_1(x) = H(x, 1)$ is the retraction map $\rho_{\model}$ and for all $t \in [0, 1]$, $H(x, t) = x$ for $x \in \sk{\model}$.
\end{prop}

Hence, for an snc model $\model$, the homotopy type of $\anl{X}$ is the same as that of the dual complex of $\model_s$, which admits a simplicial structure and is hence easier to analyse.
The proof of the prior theorem requires machinery which is outside of the scope of this report; a construction of the deformation retraction in a concrete example may be found in \parencite[\S 2.5]{birgeom}.

\section{The Structure of $k$-Analytic Spaces}

The primary goal of this section will be to prove \cref{thm:homeomorphism}, which essentially states that an analytic space $\anl X$ can be recovered by taking an inverse limit of skeleta.
In order for this result to hold, we assume that $X$ satisfies the conditions of the ambient scheme $W$ in our statement of resolution of singularities (\cref{resofsing}).

A key result in proving this is the fact that for any two distinct points $x$ and $y$ in $\anl X$, there exists an snc model $\model$ such that $\rho_{\model}(x) \neq \rho_{\model}(y)$. 
Before giving a proof of this theorem, we illustrate the strategy using the example of the analytic projective line.
To begin, $\model = \mathbb{P}^{1}_{R}$ is an snc model for $\projlinean$. 
The special fiber is simply the projective line over $\tilde{k}$ and the corresponding skeleta consists of a single type II point, corresponding to the norm on $k\{T\}$.
Consequently, we see that every point retracts to the same point. 

We split our analysis into two cases, depending on whether $\red{\model}{x}$ and $\red{\model}{y}$ are equal. 
In the first case, let $x$ be the point given by the seminorm $|f(T)|_x = |f(0)|$ for all $f \in k[T]$, and $y$ the point given by the seminorm $|f(T)|_x = |f(1)|$.
Both of these points lie in the closed disk $E(0, 1) = \berkspec{k\{T\}} \subset \projlinean$, so that the affine chart $\spec R[T] \subset \model$ contains $\red{\model}{x}$ and $\red{\model}{y}$.
Then, observe that $\resfield^{\circ\circ} \cap R[T]$ is the ideal $(\pi, T)$: if $f(T) = \pi \cdot g(T) + T \cdot h(T)$, then we may assume that $g(T) = c$ is a constant and compute:
\[
|f(T)|_x \leq \max\{ |\pi \cdot c|_x, |T \cdot h(T)|_x \} < 1.
\]
Conversely, if $f(T) = a_0 + \dots + a_n T^n$ is such that $|f(T)|_x < 1$, then: $|f(T)|_x = |a_0 + \dots + a_n T^n|_x = |a_0| < 1$, hence it follows that $a_0 \in (\pi)$ and hence $f(T) \in (\pi, T)$. 
Therefore, $\red{\model}{x}$ is the point of $\spec R[T]$ corresponding to the maximal ideal $(\pi, T)$.
A similar argument shows that $\red{\model}{y}$ is the point corresponding to the maximal ideal $(\pi, T - 1)$, and hence we are in the case $\red{\model}{x} \neq \red{\model}{y}$. 
It follows from \cref{lemma:blowupsnc} that performing a blow-up of $\model$ at either of the reduction points will give rise to an snc model $\model^{\prime}$; the claim is that $\rho_{\model^\prime}(x) \neq \rho_{\model^{\prime}}(y)$.

Choosing the point given by $(\pi, T)$ as the center of the blow-up, we find that the resulting scheme 
\[\model^\prime = \proj R[T][A, B]/(\pi B - T A)\] 
admits a covering by affine charts $U_{B} = \spec R[T, a]/(\pi - T a)$ and $U_{A} = \spec R[T, b]/(\pi b - T) \cong \spec R[b]$.
These affine charts are glued along the isomorphism \[\spec R[T, a, a^{-1}]/(\pi - T a) \cong \spec R[T, b, b^{-1}]/(\pi b - T)\] induced by the ring homomorphism sending $a \mapsto b^{-1}$.
Consequently, the special fiber is now given by the charts $\tilde{U}_{B} = \spec \tilde{k}[T, a]/(T a)$ and $\tilde{U}_{A} = \spec \tilde{k}[T, b]/(t) \cong \spec \tilde{k}[b]$ with the corresponding gluing. 
Geometrically, the blow-up has the effect of attaching a projective line to the origin of the affine line over $\tilde{k}$.
In particular, $\red{\model^{\prime}}{x}$ is now the point at infinity of the projective line in the special fiber.
Since the sets of irreducible components of the special fiber passing through $\red{\model'}{x}$ and $\red{\model'}{y}$ are disjoint, it follows that the points $\rho_{\model^\prime}(x)$ and $\rho_{\model^\prime}(y)$ are distinct.

We digress momentarily to analyse the generic fiber of $\model^\prime$.
The space $\hat{\model^{\prime}}_{\eta}$ is now formed by gluing $\berkspec{k\{T, a\}/(\pi - a T)}$ and $\berkspec{k\{T, b\}/(\pi b - t}$ along $\berkspec{k\{\pi^{-1} T, \pi T^{-1}\}}$.
Observe that the former affinoid space is an annulus with inner radius $\pi$ and outer radius $1$, while the latter is isomorphic to the closed disk $\berkspec{k\{b\}}$, which embeds into $\berkspec{k\{T\}}$ using the map $b \mapsto \pi^{-1} T$. Hence, it is identified with the closed disk of radius $\pi$, and the space $\hat{\model}^{\prime}_{\eta}$ is isomorphic to the closed unit disk $\berkspec{k\{T\}}$.

Now we consider a case where the specializations $\red{\model}{x}$ and $\red{\model}{y}$ are equal. 
For this, let $x$ be as before, and let $y$ be the point given by the norm on $k\{r^{-1} T\}$, where $r < 1$.
The earlier argument which showed that $\red{\model}{x}$ is the point of $\spec R[T]$ given by the ideal $(\pi, T)$ generalises to show that $\red{\model}{y}$ is also the point $(\pi, T)$.
Intuitively, we can think of this as a consequence of the fact that both $x$ and $y$ lie in the open disk $D(0, 1)$.

The goal now is to find a suitable center for a blow-up such that the reductions of the two points are distinct, hence reducing to the previous case.

Note that the regular function $f(T) = T + \pi^2$ is such that $|f|_x = |\pi^{2}| \neq |\pi| = |f|_y$.
Consider the subscheme $Z$ of $\model$ defined by the ideal $(\pi, T + \pi^2) = (\pi, T)$.
Blowing-up gives the same scheme computed above, but we note that both reduction points now lie in the chart $U_A$.
In particular, the points define multiplicative seminorms on the ring $R[T, b]/(\pi b - T)$, and $f/\pi = b + \pi$ is a regular function on the chart $U_A$.
We find that $|f/\pi|_x = |b + \pi|_x = |\pi|$, while $|b + \pi|_y = 1$, so it follows that the reduction points are now distinct.

To generalise these arguments, we must consider which aspects of the example above resulted in simplification which may in general be unreasonable to expect.

In the first case of our analysis, when the analytic space is not a curve, it may not be the case that one of the reduction points is closed.
In this case, we may try to take the closure of one of the reduction points as a center for the blow-up, but this center may contain singularities, so that the blow-up may not even be regular.
Consequently, to produce the desired snc model, we may invoke resolution of singularities.
If $\Pi: \model' \to \model$ is as in the statement of the theorem, then $\Pi^{-1}(\model_s) \cong \model'_s$.

In the second case, we wish to more generally find a function $f$ such that $|f|_x \neq |f|_y$ and either $|f|_x$ or $|f|_y$ is equal to $|\pi^{k}|$ for some $k$, so that we can then take the blow-up with center defined by the ideal $(\pi^{k}, f)$ in some neighbourhood of the reduction point.
Once again, the closure of the subscheme defined by this ideal may not be regular, leading us to further invoke resolution of singularities.

Equipped with these ideas, we formulate the following series of lemmas.


\begin{lemma}\label{rednotequal}
    Let $\model$ be an snc model and $x, y \in \hat{\model}_{\eta}$ distinct points such that $\rho_{\model}(x) = \rho_{\model}(y)$ but $\red{\model}{x} \neq \red{\model}{y}$.
    Then there exists a snc model $\model' \to \model$ such that $\rho_{\model'}(x) \neq \rho_{\model'}(y)$
\end{lemma}
\begin{proof}
    We may assume firstly that $\red{\model}{x} \not\in \overline{\{\red{\model}{y}\}}$. 
    Indeed, if we have both $\red{\model}{x} \in \overline{\{\red{\model}{y}\}}$ and $\red{\model}{y} \in \overline{\{\red{\model}{x}\}}$, then it follows that $\overline{\{\red{\model}{x}\}} = \overline{\{\red{\model}{y}\}} =: Z$.
    Hence, $\red{\model}{x} = \red{\model}{y}$ is the generic point of the irreducible closed subset $Z$, which is a contradiction.
    
    Now denoting $Z := \overline{\{\red{\model}{y}\}}$, we consider the blow-up $\model'' \to \model$ with center $Z$.
    In general, the subscheme $Z$ is not regular and hence $\model''$ is not guaranteed to be an snc model.
    We now perform a resolution of singularities of the pair $(\model'', \model_s'')$, resulting in a regular scheme $\model' \to \model''$.
    Since a blow-up is an isomorphism outside of its center, any singular points are contained within $\model_s''$.
    In particular, performing resolution of singularities induces an isomorphism of generic fibers $\anl{(\model_s')} \cong \anl{(\model_s'')}$.
    Furthermore, the special fiber of $\model'$ is given by the total transform of the special fiber of $\model''$.
    Hence, we see that $\model'$ is an snc model.
    
    Let $\Pi: \model' \to \model'' \to \model$ be the composition.
    We observe that $\red{\model'}{y}$ lies in $\Pi^{-1}(Z)$; consequently, the set of irreducible components of $\model'_{s}$ passing through $\red{\model'}{y}$ is distinct from those passing through $\red{\model''}{x}$, showing that ${\rho_{\model''}(x)} \neq {\rho_{\model''}(y)}$.
\end{proof}

\begin{lemma} \label{lemma:appropriatef}
    Let $A$ be an $R$-algebra of finite type, $X = \spec A$ and $x, y \in \hat{X}_{\eta}$ distinct points.
    Then, there exists a regular function $f \in A$ such that $|f|_x \neq |f|_y$ and either $|f|_x = |\pi^n|$ or $|f|_y = |\pi^n|$ for some $n$.
\end{lemma}
\begin{proof}
    If $x$ and $y$ are distinct points of $\hat{X}_{\eta} \subset \anl{X_k}$, then there exists a regular function $g \in A$ such that $|g|_x \neq |g|_y$.
    Without loss of generality, assume that $|g|_x < |g|_y \leq 1$ and that neither $|g|_x$ nor $|g|_y$ are equal to $|\pi^n|$ for any $n$.
    Consider that if there exists $n$ such that $|g|_x < |\pi^n| < |g|_y$, then $|g + \pi^n|_x = |\pi^n| < |g + \pi^n|_y = |g|_y$ by the non-Archimedean triangle inequality, so that we can set $f = g + \pi^n$ to conclude.
    In general, denote $|\pi| = p, |g|_x = r, |g|_y = s$. Then it is a matter of computing: there exists a rational $a/b$ such that $\log_{p} r > a/b > \log_{p} s$, so $\log_{p} r^b > a > \log_{p} s^b$ and hence we find that $r^b < p^a < s^b$.
    Therefore, letting $g' = g^b$, we reduce to the situation above.
\end{proof}

\begin{lemma}\label{redequal}
    Let $\model$ be an snc model and $x, y \in \hat{\model}_{\eta}$ distinct points such that $\red{\model}{x} = \red{\model}{y}$. 
    Then there exists an snc model $\model' \to \model$ such that $\red{\model'}{x} \neq \red{\model'}{y}$.
\end{lemma}
\begin{proof}
    Let $U = \spec A \subset \model$ be an affine neighbourhood of $\red{\model}{x} = \red{\model}{y}$; then, $x$ and $y$ are points in $\hat{U}_{\eta}$. 
    By \cref{lemma:appropriatef} we may assume that there exists some $f \in A$ such that $|f|_x \neq |f|_y$ and $|f|_x = |\pi^m|$ for some $m$.
    Assume that $|f|_x < |f|_y$; the other case is similar.
    Consider the closed subscheme of $U$ cut out by the ideal $(f, \pi^{m})$ and let $Z$ be its closure in $\model$. Let $\Pi: \model'' \to \model$ be the blow-up with center $Z$.
    We note that $\red{\model''}{x}$ and $\red{\model''}{y}$ are both contained in $\Pi^{-1}(U)$. 
    By considering the usual affine cover of the blow-up, we see that $\red{\model''}{x}$ and $\red{\model''}{y}$ lie in a common affine chart where the function $t = \pi^m/f$ is regular.
    We now have that $|t|_x = |\pi^{m}|/|f|_x = 1$ while $|t|_y = |\pi^{m}|/|f|_y < 1$. 
    Since the blow-up may be computed locally, we find that $\model''$ gives an $R$-model where the reduction points of $x$ and $y$ are distinct.
    In general $X'$ is not an snc model since $Z$ may not be regular, but once again, resolving singularities gives an snc model $\model' \to \model''$, and we see that $\red{\model'}{x} \neq \red{\model'}{x}$.
\end{proof}

The following theorem is now immediately implied by the existence of an snc model for $X$ and \cref{rednotequal,redequal}.

\begin{theorem} \label{injectivity}
    Let $x, y \in \anl{X}$ be distinct points. 
    Then there exists an snc model $\model$ of $X$ such that $\rho_{\model}(x) \neq \rho_{\model}(y)$.
    \qed
\end{theorem}

It is worth exploring the situations where resolution of singularities, which is an involved and intricate result, is not required.
In the above proofs, resolution of singularities was invoked twice.
\Cref{lemma:blowupsnc} showed that if the center of the blow-up is closed, then resolution of singularities does not need to be invoked, and in the case of curves we find that if two points retract to the same point of the skeleton but have different reductions, then one of the reduction points must be closed.
Indeed, if neither reduction point is closed, then they are both generic points of some irreducible component of the special fiber of the model; since they retract to the same point on the skeleton, it follows that both points are the generic point of the same irreducible component. 
Hence, for a curve the first instance of resolution of singularities is unnecessary. 
We now claim that the second invocation is also unnecessary in the case of curves.

\begin{theorem}\label{injectivityforcurves}
    Let $X$ be a curve, $x$ and $y$ distinct points of $\anl X$ and $\model$ an snc model of $X$ such that $\red{\model}{x} = \red{\model}{y}$.
    Then there exists a finite sequence of blow-ups
    \[
        \model' = \model_{n} \to \dots \to \model_{0} = \model
    \]
    where each blow-up $\model_{i + 1} \to \model_{i}$ has a center given by a closed point and each $\model_{i}$ is an snc model, such that $\rho_{\model'}(x) \neq \rho_{\model'}(y)$.
\end{theorem}

Firstly, we show this in the case where one of the points has non-trivial kernel, in other words, $x \in \hat{\model}_{\eta}$ is such that $\iota(x)$ is a closed point of $X \cong \model_k$.
In this case, we may consider the closure $F = \overline{\{\iota(x)\}}$ in $\model$.
Recall that there exists a unique map $\phi: \spec \resfield^{\circ} \to \model'$ extending the morphism $\spec \resfield \to \model'$, where the generic point of $\spec \resfield$ is mapped to the generic point of $F$, and the closed point of $\spec \resfield$ is mapped to $\red{\model'}{p}$.
The residue field at $\iota(x)$ is a finite field extension of $k$, and it follows that its completion $\resfield$ is also a finite field extension of $k$.
Consequently, the morphism $\spec \resfieldof{p}^{\circ} \to \spec R$ is finite.
In particular, $\spec \resfieldof{p}^{\circ}$ is a proper $R$-scheme.
It follows that the image of $\phi$ is the closed subscheme of $\model$ corresponding to $F$, and in particular, the point $\red{\model}{p}$ is the unique point contained in $\model_s \cap \operatorname{Supp} F$.
We note that $\red{\model}{x}$ is then a closed point in $\model$, so taking the blow-up with center $\red{\model}{x}$ results in an snc model.

\begin{lemma}\label{injectivityforzariskiclosed}
    Let $X$ be a curve, $x$ and $y$ distinct points of $\anl{X}$ and $\model$ an snc model of $X$ such that $\red{\model}{x} = \red{\model}{y}$.
    Assume that $\iota: \anl{X} \to X$ maps $x$ to a closed point of $X$.
    Then there exists a finite sequence of blow-ups
    \[
        \model' = \model_n \to \dots \to \model_0 = \model
    \]
    where each blow-up $\model_{i + 1} \to \model_i$ has center given by a closed point, each $\model_i$ is an snc model, $\rho_{\model'}(x) \neq \rho_{\model'}(y)$ and, additionally, $\red{\model'}{x} \neq \red{\model'}{y}$.
\end{lemma}
\begin{proof}
    It is notationally convenient to switch to additive notation using semivaluations instead of seminorms. If $\model_i$ is an snc model as in the statement of the theorem, denote the corresponding reduction map by $\operatorname{red}^{i}$ and the corresponding retraction map by $\rho^{i}$.
    
    Firstly, we note that since $\resfield$ is a finite field extension of $k$, it follows that $|k^{\times}|$ has finite index in $|\resfield^\times|$, so that $|\resfield^{\times}|/|k^{\times}|$ is a finite group.
    Denote by $b(x)$ the order of this group.
    In particular then, for any $f \in \sheaf_{\model, \red{\model}{x}}$ which is not a unit, $v_x(f) = a(f)/b(x)$ for some integer $a(f) \geq 1$.
    
    Let $f \in \sheaf_{\model, \red{\model}{x}}$ be such that $v_x(f) \neq v_y(f)$.
    Assume that we have constructed a sequence of blow-ups $\model_{i} \to \dots \to \model_{0} = \model$, for $i \geq 0$, as in the statement of the theorem.
    If $\redi{i}{x} \neq \redi{i}{y}$, then we can conclude using \cref{rednotequal}.
    So assume that $\redi{i}{x} = \redi{i}{y} =: z$, so that $z$ is a closed point of $\model_i$.
    Let $z_1, z_2$ be an snc system of parameters at $z$, and let $\model_{i + 1} \to \model_{i}$ be the blow-up with center $z$.
    
    \paragraph{Case 1}
    If $v_y(z_1) < v_y(z_2)$, then it follows that $\redi{i + 1}{x} \neq \redi{i + 1}{y}$.
    Indeed, assume for a contradiction that the reduction points are equal.
    Then, either $z_1/z_2$ or $z_2/z_1$ is regular on a neighbourhood of $\redi{i + 1}{x}$; assuming that it is the former, we find that $v_x(z_1/z_2) < 0$, which gives a contradiction.
    We may now conclude by \cref{rednotequal}.
    
    \paragraph{Case 2}
    
    Assume that $v_y(z_1) \geq v_y(z_2)$, so that $\redi{i + 1}{x} = \redi{i + 1}{y}$.
    Furthermore, suppose that $v_x(z_2) \neq v_y(z_2)$.
    Since $\rho^{i}(x) = \rho^{i}(y)$, it follows from construction of the retraction map that we must have $v_x(z_1) = v_y(z_1)$.
    Then, $z_2, z_1/z_2$ is a regular system of parameters at $\redi{i+1}{x}$, but we now note that $v_y(z_2) \neq v_x(z_2)$ and $v_y(z_1/z_2) \neq v_x(z_1/z_2)$, so it follows that $\rho^{i + 1}(x) \neq \rho^{i + 1}(y)$.
    
    If $\redi{i + 1}{x} = \redi{i + 1}{y}$, then we note that since $x$ has non-trivial kernel, it cannot be a point on $\sk{\model_{i + 1}}$.
    Hence, $\rho^{i + 1}(x)$ is a divisorial point by \cref{lemma:retractiondivisorial}, so that by taking finitely many blow-ups, where each center is a closed point, we may find a model $\model'$ with respect to which $\rho^{i + 1}(x)$  is divisorial.
    It then follows that $\red{\model'}{x} \neq \red{\model'}{y}$.
    
    \paragraph{Case 3}
    
    We may now assume that $v_x(z_2) = v_y(z_2)$.
    Writing  $f = z_1 f_1 + z_2 f_2$, we see that in $\sheaf_{\model_{i + 1}, \redi{i + 1}{x}}$, $f$ can be written in the following form:
    \begin{align*}
        f = z_2 \cdot \frac{z_1}{z_2} f_1 + z_2 f_2 = z_2 \left( \frac{z_1}{z_2} \cdot f_1 + f_2 \right) = z_2 f'.
    \end{align*}
    Then, $v_x(f') = v_x(f) - v_x(z_2)$, and $v_x(z_2) \geq 1/b(x)$.
    Since $v_y(z_2) = v_x(z_2)$, we see that $v_y(f') \neq v_x(f')$.
    Iterating this process, replacing $\model_i$ with $\model_{i + 1}$ and $f$ with $f'$, we see that this procedure terminates after finitely many steps.
    Then, $\red{\model''}{x} \neq \red{\model''}{y}$, and we again conclude by \cref{rednotequal}.
\end{proof}

For any $f \in K(X)^{\times}$, a point $x \in \anl{X}$ is called a zero of $f$ if $|f|_x = 0$.
For any closed point $p$ of $X$, the residue field $\kappa(p)$ is a finite field extension of $k$, and admits a unique non-Archimedean absolute value extending the absolute value on $k$.
There is hence a unique point $x \in \anl{X}$ where $\iota(x) = p$.
Since $X$ is a proper, integral scheme of dimension $1$, we see that a rational function $f$ has finitely many zeros. 
This observation may now be used to generalise the prior lemma to any set of points.

\begin{proof}[Proof of \cref{injectivityforcurves}]
    Fix an element $f \in \sheaf_{\model, \red{\model}{x}}$ such that $v_x(f) \neq v_y(f)$.
    By \cref{injectivityforzariskiclosed}, we may find a proper morphism $h: \model' \to \model$ such that the zeros of $f$ reduce to distinct points on $\model'_s$, and by applying \cref{injectivityforzariskiclosed} again, we can further assume that $\red{\model'}{x}$ and $\red{\model'}{y}$ are distinct from the reductions of the zeros of $f$.
    
    Let $D = \divisor_{\model'}(f)$ be the divisor of $f$ on $\model'$.
    Since $f$ is regular on a neighbourhood $U$ of $\red{\model}{x}$, the intersection $D \cap h^{-1}(U)$ is an effective divisor on $h^{-1}(U)$, and we may hence assume that $D$ is effective.
    There is a decomposition $D = \tilde{D} + E$, where the prime components of $E$ are prime components which appear in the special fiber $\model'_s$.
    Let $F$ be a prime component of $\tilde{D}$.
    The generic point of $F$ corresponds to a closed point of the generic fiber $\model'_k$ and hence to a zero $p \in \anl{X}$ of $f$, and by our earlier remarks, the only point contained in $\operatorname{Supp}(F) \cap \model'_s$ is then $\red{\model'}{p}$.
    
    In particular, we find that $\red{\model'}{x}$ and $\red{\model'}{y}$ are not contained in $\operatorname{Supp}(\tilde{D})$.
    Hence, $v_x(f) = v_x(\tilde{D} + E) = v_x(E)$, where the first equality is by definition of the valuation of a divisor.
    The second equality is due to the fact that we may choose a local equation $g = 0$ for $D$ such that $g$ is a unit in $\sheaf_{\red{\model'}{x}}$, and hence 
    \[v_x(\tilde{D} + E) = v_x(g \cdot g') = v_x(g) + v_x(g') = v_x(g') = v_x(E)\]
    where $g' = 0$ is a local equation for $E$.
    
    Denoting $z = \rho_{\model'}(x)$, we claim that $v_x(E) = v_z(E)$. 
    This follows from the fact that $v_x(F) = v_z(F)$ for any irreducible component of $\model_s$ passing through $\red{\model'}{x}$; the multiplicity with which each such $F$ appears in $E$ then completely determines $v_x(E)$ and shows that $v_x(E) = v_z(E)$.
    The same argument then shows that the corresponding equation holds for $y$, so that $\rho_{\model'}(x) \neq \rho_{\model'}(y)$ since $v_x(E) \neq v_y(E)$.
\end{proof}

\Cref{retractionsinversesystem} shows that proper morphisms of snc models $\model' \to \model$ induce an inverse system of skeleta, composed of the retraction maps $\rho_{\model}: \sk{\model'} \to \sk{\model}$.
The following theorem then says that taking the limit over all skeleta recovers the space $\anl{X}$, providing a connection between the geometry of $\anl{X}$ and the birational geometry of $X$.
This result is stated without proof in \parencite[p. 77, Theorem 10]{kontsevich}, and a proof is given using significantly more advanced and vastly different techniques in algebraic geometry in \parencite[Corollary 3.2]{bfj}.
We stress that the arguments we have presented in this chapter are not simplifications of those found in \textit{loc. cit.} and have been constructed independently.
A proof is also provided in the algebraically closed case for curves in \parencite[Theorem 5.2]{bpr}.
There are several results in the literature investigating how an analytic space may be expressed as an inverse limit of simplicial complexes and similar polyhedral topological spaces.
A set of references to these results may be found in the discussion preceding \parencite[Corollary 3.2]{bfj}.

\begin{theorem}\label{thm:homeomorphism}
    There is a homeomorphism
    \[
        \anl X \cong \varprojlim \operatorname{Sk}(\model)
    \]
    where $\model$ ranges over the snc models of $X$.
\end{theorem}

\begin{proof}
    The continuous maps $\rho_{\model}: \anl X \to \sk{\model}$ are such that $\rho_{\model} \circ \rho_{\model'} = \rho_{\model}$, so that by the universal property of the limit, there exists a unique map $u: \anl X \to \varprojlim \sk{\model}$.
    Since $X$ is proper, $\anl X$ is compact. 
    Furthermore, each skeleton $\sk{\model}$ is compact and Hausdorff, so that $\varprojlim \sk{\model}$ is also compact and Hausdorff.
    As a result, it suffices to show that the map $u$ is a bijection.
    It follows by \parencite[\S 9.6, Cor. 2]{bourbaki} that since each map $\rho_{\model}$ is surjective, the map $u$ is also surjective.
    Finally, injectivity is a direct consequence of \cref{injectivity}.
\end{proof}

An application of this theorem may be found in \parencite{bfj}; we suppress further discussion as it falls outside of the scope of this project.
